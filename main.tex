\documentclass[12pt]{report}

\usepackage[utf8]{inputenc}
\usepackage[top=1in, bottom=1in, left=1in, right=1in]{geometry}
\usepackage{fancyhdr}
\usepackage{hyperref}
\usepackage{listings}
\usepackage{graphicx}
\usepackage{float}
\usepackage{color}
\usepackage{courier}

\pagestyle{fancy}
\graphicspath{{images/}}

\title{Programming in Meruem}
\author{Melvic C. Ybanez}
\date{September 2015}

\hypersetup{hidelinks}

\newcommand{\keyword}[1]{\textit{#1}}
\newcommand{\ikeyword}[1]{\textbf{#1}}
\newcommand{\code}[1]{\texttt{#1}}
\newcommand{\codexleftmargin}{8.0ex}
\newcommand{\codenumbersep}{14pt}
\newcommand{\codenumberstyle}{\small\color{gray}}
\newcommand{\codecommentstyle}{\itshape\color{gray}}
\newcommand{\codebasicstyle}{\ttfamily}
\newcommand{\sectionref}{\ref}

\definecolor{gray}{rgb}{0.4,0.4,0.4}

\lstnewenvironment{Python}{
	\lstset{
		language=Python, 
		basicstyle=\codebasicstyle,
		numbers=left, 
		numbersep=\codenumbersep,
		showstringspaces=false,
		tabsize=2,
		xleftmargin=\codexleftmargin,
		escapeinside={(*@}{@*)},
		breaklines=true,
		numberstyle=\codenumberstyle,
		commentstyle=\codecommentstyle
	}
}{}
\lstnewenvironment{Java}[1][1]{
	\lstset{
		language=Java,
		basicstyle=\codebasicstyle,
		numbers=left,
		numbersep=\codenumbersep,
		showstringspaces=false,
		tabsize=4,
		xleftmargin=\codexleftmargin,
		escapeinside={(*@}{@*)},
		firstnumber=#1,
		breaklines=true,
		numberstyle=\codenumberstyle,
		commentstyle=\codecommentstyle
	}
}{}
\lstnewenvironment{REPL}{
	\lstset{
		language=Lisp, 
		basicstyle=\codebasicstyle,
		xleftmargin=\codexleftmargin,
		breaklines=true,
		commentstyle=\codecommentstyle,
		otherkeywords={meruem>},
		showstringspaces=false
	}
}{}
\lstnewenvironment{Meruem}[1][1]{
	\lstset{
		language=Lisp,
		basicstyle=\codebasicstyle,
		numbers=left,
		numbersep=\codenumbersep,
		xleftmargin=\codexleftmargin,
		showstringspaces=false,
		tabsize=2,
		escapeinside={(*@}{@*)},
		breaklines=true,
		numberstyle=\codenumberstyle,
		commentstyle=\codecommentstyle
	}
}{}

\newenvironment{noteparagraph}[1]{
	\bigskip
	\small{\textbf{Note}: #1}
}{\par}

\begin{document}
	\maketitle
	
	\tableofcontents
	
	\chapter*{Introduction}
	\addcontentsline{toc}{chapter}{Introduction}
	\section{About the book}
This book is a tutorial for the Meruem programming language, written by the people who developed the current version of Meruem. Our goal is to teach you the introductory concepts of functional programming and (to some extent) metaprogramming using the Meruem language. 

\section{Who should read this book}
If you are someone looking for the next popular object-oriented programming language to master and doesn't feel like learning new and more mathematical ways of solving problems for now, then this tutorial is not for you. There are many options for new such languages out there, but Meruem is not (yet) one of them.

That said, if you are someone willing to spend a lot of time mastering not just a new programming language but also different programming paradigms, hoping that you will be able to apply all the knowledge you can gain from this book with any other programming languages you already know, then this book is for you.

\section{Programming background required}
This book is written primarily for imperative and/or object-oriented programmers who want to learn functional programming and metaprogramming, or people who don't know programming at all. If you are already familiar with functional programming and metaprogramming, then most of the contents here will not be new to you, but this book can still serve as a review material.

\section{How to read this book}
Most of the chapters in this book are not self-containing, so I recommend you read them in the proper order starting from chapter 1, especially if you are new to Lisp-like languages like Meruem. 	
	
	\chapter{Starting Out}
	\section{What is Meruem?}
\ikeyword{Meruem} is a dynamically-typed, interpreted programming language that supports both \keyword{functional programming} and \keyword{metaprogramming}, and runs on top of the \keyword{Java Virtual Machine}(JVM).

Meruem is also a \keyword{Lisp} dialect. That means it has most, if not all, of the characteristics common to all Lisps, like \keyword{homoiconicity}, \keyword{macros}, and a small, simple and elegant core.

\section{Why learn Meruem?}
Meruem will change the way you think about programs, programming, and problems in general. The things that you will learn from this book will still be applicable to your day-to-day job as a programmer, even if you will be using a different and more mainstream programming language. This is because learning Meruem is not just learning a new programming language, it's learning completely new programming paradigms. Knowing different programming paradigms (imperative, OOP, FP, etc) is always a good thing since it would give you different ways of solving problems. After you've learned Meruem, you'd realize that there's more to programming than just \keyword{imperative programming}.

\section{Installing Meruem}
To program in Meruem, you need to install Java and download the Meruem interpreter.

\subsection{The Java Virtual Machine}
As I've said above, Meruem runs on the Java platform, which is a JVM (sometimes I just refer to it as "the JVM"). To be more accurate, the current version of Meruem actually gets ran by the Scala programming language, which runs on top of the JVM. What I mean by that is that the interpreter of Meruem is written in Scala. 

But, just what is a JVM? 

According to Wikipedia, a JVM is "an abstract computing machine that enables a computer to run a Java program". Essentially, without a JVM, we can't run Java programs. 

So how do Scala programs run on it if it only understands Java bytecode? Simple, the Scala compiler generates Java bytecode. And since Meruem is written in Scala, then a Meruem code will eventually be converted to Java bytecode. 

So we need to install a JVM in order to run our Meruem interpreter. To do that, we install a \keyword{Java Runtime Environtment}(JRE). Installing a JRE was what I meant earlier by installing Java. A JRE contains the JVM, libraries, and some other things we shouldn't worry about in this book. There are many instructions on the web on how to install a Java runtime environment on different platforms, such as this one: \url{ https://www.java.com/en/download/help/download_options.xml}

Note: There is also what is known as a \keyword{Java Development Kit}(JVM). You have to install it if you want to develop Java programs and not just being able to run them. A JVM already contains a JRE so you don't need to install both.

\subsection{Downloading the interpreter}
When you installed the JRE, you've already installed the JVM as well. . However, the JVM alone is not enough to run Meruem programs. That's because it only understands Java bytecodes


	
	\chapter{Data Types}
	In this chapter, we are going to make use of the REPL only. We are not going to use Winter (or any text editor you have right now). If you haven't already installed the REPL, please go back to section~\ref{sec:installing-meruem} and follow the instructions on how to install Meruem before proceeding. Remember, the Meruem distribution is already bundled with a REPL.

\section{Data Types}
Programming always involves manipulating data. For instance, writing a program that adds two random numbers involves working on numbers. Reading the contents of a file involves the manipulation of files and strings. An enrolment system requires the presence of data that represent the student information, the class schedules, and others. Even the \code{Hello World} program that we wrote earlier wouldn't even be completed if we didn't know what data to print to the screen. Whatever it is you want to do, you need some data.

Now, the thing about data is they don't all have the same classifications, and the operations that you can perform on a data depend on the classification of that data. For example, you can add a number to another number but you can't add a number to a student information. (That wouldn't really make sense.) This classification of data is known as a \keyword{data type}.

A \ikeyword{data type} tells you how a thing is classified, what set of values belongs to this type, and what operations can be performed on it. Meruem has a short list of supported data types. Let's discuss each of them, starting with the \code{Number} types.
\end{document}