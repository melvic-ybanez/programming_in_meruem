Meruem is a functional programming language. So when you program in Meruem, you don't only rely on the power of lists, you rely heavily on the power of functions too. Creating and using functions are one of the most common tasks in functional programming, so we better know how to do both. In this chapter we are going to learn how to use some of the existing standard functions of the Meruem programming language.

\section{What is a Function?}
A \ikeyword{functoin} is piece of code that is designed to perform a specific computation. A function takes a set of inputs (which may be empty), does something with it, and return an output. The inputs are known as \keyword{parameters} and the output is known as the \keyword{return value}. In the previous chapter, you've learned that the syntax of a function call looks like this:

\begin{QuasiLang}
(function arg1 arg2 arg3 ...)
\end{QuasiLang}

Note that I've replaced \code{callable} with \code{function}, to emphasize that most of the callables are effectively just functions. The elements starting from the second one are the \keyword{arguments} of the function. Function arguments are additional data that you pass to the function in order for it to complete its computations and return the result. The function receives the values passed as arguments by storing them to \keyword{parameters}.

\section{Evaluating a Function}
In order to evaluate a function call, the arguments need to be evaluated first. Then, you apply the function (first element) to the evaluated arguments. The result of evaluating the first argument gets passed to the first parameter of the function, the result of evaluating the second argument gets passed to the second parameter, and so on. Since function calls are s-expressions, and s-expressions are recursive in nature, some of the arguments of a function can be complex expressions or function calls too. 
\\~\\
\textbf{Note}: There are some functions that don't evaluate all of their arguments first before evaluating the function itself. You will learn about them later.

Lists are normally just evaluated as function calls. The expression \code{(1 3 4)} is not executable, because while it is a list, its first element is not a function. If you want to call a function that doesn't take any arguments, just wrap the function in parentheses, effectively creating a list with only one element (e.g. \code{(some-function)}).

Now that you know what functions are and how function evaluation works, it's time to test some standard and built-in functions of Meruem.

\section{Type-converting Functions}
In many cases, it is very useful to convert one type to another. For that reason, Meruem provides some type converting functions that do exact that kind of conversion:

\begin{REPL}
meruem> (to-long 4)
4
meruem> (to-int 30.23)  
30
meruem> (to-double 67)
67.0
meruem> (to-float 45@l)
45.0
meruem> (to-string 56)
"56"
meruem> (to-string 56@d)
"56.0"
meruem> (to-char 45)
\-
meruem> (to-char 97)
\a
\end{REPL}

The names of these functions are self-explanatory. For instance, \code{to-long} takes a number and tries to convert it to a \keyword{long}. If you apply a function to argument whose type is not compatible with the type the function is expecting, you'll get an error:

\begin{REPL}
meruem> (to-char "the")
An error has occurred. Invalid Type. Not a Number: the
Source: .home.melvic.meruem.meruem.prelude [1:10}]
(to-char "the")
         ^
\end{REPL}

In this example, the interpreter was expecting a number, and you gave it a string. If you are not sure what type an expression has, you can use the \code{type} function to retrieve it:

\begin{REPL}
meruem> (type 20)
Integer
meruem> (type 45.456)  ; double, even without the @d
Double
\end{REPL}

\section{Functions as Operators}
As mentioned in the previous chapter, operators in Meruem are actually just functions whose names are composed of operator (or special) symbols. In other words, there is nothing special about them. In the following subsections you are going to see the different operators supported by Meruem.

\subsection{Arightmetic Operators}
\ikeyword{Arightmetic operators} are the basic mathematical operators that we have learned since primary school. Let's review them using s-expressions:

\begin{REPL}
meruem> (+ 1 2 3)
6
meruem> (- 56 78)
-22
meruem> (* 34 2 3 5)
1020
meruem> (/ 100 20)
5
meruem> (% 19 4)
3
\end{REPL}

The only operator there not that was never taught in school was the last one (\code{\%}). It is known as the \keyword{modulus} operator. The modulus operator will return the remainder after dividing the first operand with the second one.

Each of these operators or functions receives a varying number of arguments. For instance, the \code{+} operator can be applied to more any number of arguments you want:

\begin{REPL}
meruem> (+ 34 65 6 7 87 5 4 3)
211
meruem> (+ 4 5)
9
\end{REPL}

What if you don't pass any arguments at all? Well, the addition function returns the additive identity number, the multiplication function returns the multiplicative identity number, and the substraction and division operators return errors:

\begin{REPL}
meruem> (+)
0
meruem> (*)
1
meruem> (/)
An error has occurred. Incorrect number of arguments: 0
Source: .home.melvic.meruem.meruem.prelude [9:28}]
(defun lazy (expr) (lambda () ,expr))

                           ^
meruem> (-)
An error has occurred. Incorrect number of arguments: 0
Source: .home.melvic.meruem.meruem.prelude [9:28}]
(defun lazy (expr) (lambda () ,expr))

                           ^
\end{REPL}

Since they are all just s-expressions, you can nest them to your heart's content:

\begin{REPL}
meruem> (+ 5 (- 788 89 0) 54  9 (/ 4 5 (* 4 6)))
767
\end{REPL}

You can create the most complex mathematical expressions using only these simple contructs. Now, take a lot at this example:

\begin{REPL}
meruem> (/ 100 7)
14
\end{REPL}

We all know that 100 is not divisible by 7, so why did the expression return a whole number? The reason is that the interpreter performs an \keyword{integer division}. That is, it disregards the decimal digits after the division. If you want to include the decimal digits in the result, you need to convert one of them into a float or double:

\begin{REPL}
meruem> (/ 100@f 7)
14.285714
meruem> (/ (to-double 100) 7)
14.285714285714286
\end{REPL}

\subsection{Relational Operators}
What if you want to know which of the two numerical expressions is greater? What if you want to check if the two function calls return the same values? In programming, comparing things is very common. That is why Meruem was designed to have sufficient number of \keyword{relational operators}. 

\ikeyword{Relational operators} are operators that are used to check the relationship between two or more expressions. The relational operators supported by Meruem includes the \keyword{equality} and \keyword{inequality} operators:

\begin{REPL}
meruem> (= 45 3)                
false
meruem> (> 55 7)
true
meruem> (< 345 6)
false
meruem> (>= 54 3)
true
meruem> (<= 67 45)
false
meruem> (> 5 6 6)         
false
meruem> (= () '(1 2 3))
false
meruem> (< () ())
An error has occurred. Invalid Type. Not a Number: ()
Source: .home.melvic.meruem.meruem.prelude [9:28}]
(defun lazy (expr) (lambda () ,expr))

                           ^
\end{REPL}

Notice that relational functions can take varying number of arguments too. Also, while the equality operator can take non-numeric arguments, the others can't.

\subsection{Logical Operators}
The last types of standard operators in Meruem are the \keyword{logical operators}. There are 3 logical operators supported in the language: \code{and}, \code{or}, and \code{not}. All three of them can take varying number of arguments. They are also one of those functions that don't necessarily evaluate all of their arguments prior to the actual application.

\subsubsection{And}
The \ikeyword{and} operator checks the truthfulness of all of the arguments. If an argument is falsy, it returns that argument, and the rest of the argument won't even be evaluated. If all arguments have been evaluated, it would return the last one. If no arguments are provided, the \code{and} operator returns true.

Before we jump to examples, it is worth discussing what values are considered "truthy" and what values are considered "falsy" in Meruem. Well, it's actually pretty simple. The values \code{false} and \code{nil} are considered falsy values. The rest are thruthy.

Now let's see some sample usages of the \code{and} operator:

\begin{REPL}
meruem> (and 45 false 6)
false
meruem> (and 45 nil 5)
nil
meruem> (and 45 89)
89
meruem> (and)
true
meruem> (and nil 55 3 true)
nil
\end{REPL}

\subsubsection{Or}
The \ikeyword{or} operator also checks the truthfulness of the arguments. It also behaves just like the \code{and} operator. However, unlike \code{and}, \code{or} will return the first truthy (and not the falsy) argument found. The following examples show the difference:

\begin{REPL}
meruem> (or 45 false 6)
45
meruem> (or 45 nil 5)
45
meruem> (or nil false 7)
7
meruem> (or nil 7 false)
7
meruem> (or nil false nil)  
nil
meruem> (and 1 2 false)
false
\end{REPL}

\subsubsection{Not}
The last logical operator is \ikeyword{not}. It simply negates an expression, returning the opposite of an expression's truthfulness:

\begin{REPL}
meruem> (not true)
false
meruem> (not false)
true
meruem> (not nil)
true
meruem> (not 20)
false
\end{REPL}

To determine whether a value or expression is truthy or falsy, we can use the \code{truthy?} and \code{falsy?} functions respectively.

\begin{REPL}
meruem> (truthy? true)
true
meruem> (truthy? false)
false
meruem> (truthy? nil)
false
meruem> (truthy? 1)
true
meruem> (falsy? true)
false
meruem> (falsy? 1)
false
meruem> (falsy? false)
true
\end{REPL}

\section{Conditional Functions}
Sometimes you may want to decide at runtime what code to execute (or expression to evaluate) based on a given condition. You can do this using \keyword{conditional expressions}. 

\subsection{If-expressions}
The simplest form of conditional expressions is the \keyword{if-expression}. An if-expression takes three arguments. First, it checks the truthfulness of the first one. If the first argument is truthy, it evaluates the second argument and return the result. Otherwise, it evaluates the third argument and return the result. Here's are some if-expressions in action:

\begin{REPL}
meruem> (if (> 10 4) "greater" "lesser")
"greater"
meruem> 'foo                   
foo
meruem> (if 'foo 5 7)
5
meruem> (if nil false true)
true
meruem> (if (= () ()) (if true 10 9) nil)  ; nested if-expressions
10
\end{REPL}

\subsection{The Cond Expression}
The other, more general, conditional expression is the \keyword{cond expression}. It takes a list of pairs. In each of the pairs, \code{cond} would evaluate the first element. If the first element of the pair is truthy, it would evaluate the second element, halt the evaluation process (meaning the rest of the pairs would no longer be evaluated), and returns the result of evaluating the second element. You'll understand this better with a few examples:

\begin{REPL}
meruem> (cond (true 5) (false 6))
5
meruem> (cond (nil "ignore this") (false 'nope) (5 "yaay"))
"yaay"
meruem> (cond (nil 5) (false 5))	; returns nothing
nil
meruem> (cond)  ; no conditions to check
nil
\end{REPL}

\section{Predicates}
A \ikeyword{predicate} is a function that maps an expression to a boolean value. Basically, all they do is just take an argument and check if that argument is something that falls in a certain category. Here are the standard predicates currently present in Meruem (you've already seen some of them):

\begin{description}
	\codedesitem{atom?} checks if the given argument is an atom.
	\codedesitem{symbol?} checks if the given argument is a symbol.
	\codedesitem{list?} checks if the given argument is a list.
	\codedesitem{truthy?} checks if the given argument is truthy.
	\codedesitem{falsy?} checks if the given argument is falsy.
	\codedesitem{even?} checks if the given argument is an even number. It doesn't accept non-numeric types.
	\codedesitem{odd?} checks if the given argument is an odd number. It doesn't accept non-numeric types.
	\codedesitem{empty?} checks if a string or list is empty. It doesn't accept any other types of values.
\end{description}

Each of these names is actually self explanatory. Let's see some examples of some of them:

\begin{REPL}
meruem> (atom? 6)
true
meruem> (list? ())
true
meruem> (even? 67)
false
meruem> (empty? ())
true
meruem> (empty? "hello")
false
\end{REPL}

\section{More Utility Functions}
This section will discuss a few other commonly used functions.

\subsection{Apply}
The \ikeyword{apply} function takes a function and a list and apply the function to the unpacked contents of the list as if the contents were passed directly to that function:

\begin{REPL}
meruem> (apply + '(1 2))
3
meruem> (apply list '(1 2 4))
(1 2 4)
meruem> (apply if '((< 5 6) true nil))
true
\end{REPL}

The \code{apply} function is useful when you want to apply a certain function to a set of arguments whose size and values aren't known until runtime (i.e, the arguments were dynamically generated). For instance, supposing that there is a function \code{foo} that takes an integer and returns a list of integers. What is the easiest way for you to compute their sum? You can't do \code{(+ (foo 10))} because the \code{+} function takes a list (as in a set) of integer arguments, not an argument that is a list (as in lisp list)  of integers. The way to do what you want is to use the \code{apply} function, like this: (apply + (foo 10)).

\subsection{Lazy}
The \ikeyword{lazy} function (or macro) takes a single argument, wrap this argument in a function that doesn't take any arguments, then return that function. The argument being passed to the \code{lazy} function does not get evaluated prior to the function call. The purpose of \code{lazy} is simply to "delay" the evaluation of an expression. Let's see some examples:

\begin{REPL}
meruem> (lazy 56)  ; returns a function that returns 56
<function>
meruem> ((lazy 56))  ; evaluates the function returned by lazy
56
\end{REPL}

Note that \code{<function>} is the string representation of a function. The \keyword{string representation} of an expression is the string value returned by applying \code{to-string} to that expression.

\subsection{Identity}
\label{sec:identity}
The \ikeyword{identity} function takes a single expression and immediately return that expression:

\begin{REPL}
meruem> (identity 45)
45
meruem> (identity "hello")
"hello"
meruem> (identity (+ 5 3))
8
\end{REPL}

It looks so useless, right? Well, you'll understand it's purpose when we get to \keyword{higher-order functions}.

\subsection{Increment and Decrement}
The last functions we are going to discuss in this section are the increment and decrement functions. The \ikeyword{increment} function, denoted by \code{inc}, takes a number and returns that number increased by one. It is essentially the same as \code{(+ 1 arg)}, where \code{arg} is a numeric argument. The \ikeyword{decrement} function, denoted by \code{dec}, is the opposite of the increment function (i.e it is the same as \code{(- 1 arg)}). 


Finally, here is the last set of examples for this chapter:

\begin{REPL}
meruem> (inc 67)
68
meruem> (dec 30)
29
meruem> (inc 9.8)
10.8
meruem> (inc \a)
An error has occurred. Invalid Type. Not a Number: a
Source: .home.melvic.meruem.meruem.prelude [1:6}]
(inc \a)
     ^
\end{REPL}