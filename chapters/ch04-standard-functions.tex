Meruem is a functional programming language. So when you program in Meruem, you don't only rely on the power of lists, you rely heavily on the power of functions too. Creating and using functions are one of the most common tasks in functional programming, so we better know how to do both. In this chapter we are going to learn how to use some of the existing standard functions of the Meruem programming language.

\section{What is a Function?}
A \ikeyword{functoin} is a named piece of reusable code that takes a set of inputs and return an output. The inputs are know as \keyword{parameters} and the output is known as the \keyword{return value}. In the previous chapter, you've learned that the syntax of a function call looks like this:

\begin{QuasiLang}
(function arg1 arg2 arg3 ...)
\end{QuasiLang}

Note that I've replaced \code{callable} with \code{function}, to emphasize that most of the callables are effectively just functions. The elements starting from the second one are the parameters of the function. 

\section{Evaluating a function}
In order to evaluate a function call, the arguments need to be evaluated first. Then, you apply the function (first element) to the evaluated arguments. In other words, the result of evaluating the first argument becomes the first parameter of the function, the result of evaluating the second argument becomes the second parameter, and so on. Note that some of these arguments can be function calls too.

Lists are normally just evaluated as function calls. This is why the expression \code{(1 3 4)} is not executable, because while it is a list, its first element is not a function.