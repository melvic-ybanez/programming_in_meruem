\section{What is Recursion?}
\ikeyword{Recursion} is an approach to solving a problem by using a function that is composed of simpler instances of the same function. It is essentially similar to constructing a function that calls itself. 

One common example is the recursive, mathematical definition of \keyword{factorial}. The \ikeyword{factorial} of a positive integer $n$ is the product of $n$ and the \keyword{factorial} of $1$ less than $n$. This is a recursive definition because the term being defined (which is $factorial$) appears, as an important element, in it's definition.

Here's a sample implementation of a factorial function in Meruem:

\begin{Meruem}
(defun factorial (n)
  (if (= n 0)
    1
    (* n (factorial (dec n)))))

(defun main (args)
  (println (factorial 
             (to-int (head args)))))
\end{Meruem}

Before we discuss the \code{factorial} function itself, let's first take a look at our main function here. The main function applies the \code{head} function --- which takes a list and returns its first element --- to \code{args}. Then it converts the result (which, in this case, is a string) to \code{int}. Next, it applies the \code{factorial} function to the integer result. (\code{factorial} can only be applied to integers so we had to convert the \code{head} of \code{args} to \code{int} first.) Lastly it prints the result of calling \code{factorial}. Basically, what we're doing is telling the user to pass as a command line argument the value \code{factorial} is to be applied to. With this, we can run the program multiple times with different arguments without having to modify the code.

Now, let's talk about the \code{factorial} function. It takes an integer and checks if its value is \code{0}. If it is \code{0}, then the function returns 1. Otherwise, the function returns the product of the integer \code{n} and the result of applying itself to the value of \code{n} decreased by 1. In this case, the \code{factorial} function is performing a recursive call. Let's try running it a few times to see if it will behave as we wanted it to:

\begin{REPL}
$  java -jar $MERUEM_HOME/meruem.jar factorial 5
120
$ java -jar $MERUEM_HOME/meruem.jar factorial 10
3628800
$ java -jar $MERUEM_HOME/meruem.jar factorial 4
24
\end{REPL}

In order to have an idea of what's really happening when we ran \code{factorial}, you can substitute each of the argument values for every parameter \code{n} in each call to \code{factorial}. For instance, here's what happens when you execute \code{(factorial 5)}:

\begin{flalign*}
factorial(5) &= 5 \times factorial(4) \\
&= 5 \times (4 \times factorial(3)) \\
&= 5 \times (4 \times (3 \times factorial(2))) \\
&= 5 \times (4 \times (3 \times (2 \times factorial(1)))) \\
&= 5 \times (4 \times (3 \times (2 \times (1 \times factorial(0))))) \\
&= 5 \times (4 \times (3 \times (2 \times (1 \times 1)))) \\
&= 120
\end{flalign*}

As you can see, we couldn't compute for the factorial of 5 without knowing the factorial of 4 first. And we couldn't get the factorial of 4 without calling the factorial of 3, and so on. When we finally got to $factorial(0)$ --- or \code{(factorial 0)} in Meruem --- the recursive calling stopped. That is because we hit the \keyword{base case} (I'll explain this in a minute), which says that the factorial of 0 is 1. When $factorial(0)$ returned 1, $factorial(1)$ resumed it's computation. When $factorial(1)$ completed and returned, $factorial(2)$ resumed, and so on. Eventually, the original call --- $factorial(5)$ --- returned and we had our result.

A recursive function should have a base case. A \ikeyword{base case} is the case in which the result can be computed without resorting to a recursive call. In our factorial function, the base case happens when the input is 0. You can say that $factorial(0)$ is the simplest instance of the factorial function. If we remove the base case of the factorial function, calling it would cause an \keyword{infinite recursion}, because calling $factorial(0)$ would result to $0 \times factorial(-1)$, instead of $1$, $factorial(-1)$ resolves to $-1 \times factorial(-2)$, and so on.

\section{Recursion and StackOverflowErrors}

Here's a function that adds all the positive integers from 1 up to the integer $n$: 

\begin{Meruem}
(defun sum-1-to-n (n)
  (if (<= n 0)
    0    
    (+ n (sum-1-to-n (dec n)))))
\end{Meruem}

First, it checks if the parameter is bound to \code{0} or a negative number. If it is, the function returns \code{0}. Otherwise, it returns the sum of \code{n} and the \code{sum-1-to-n} of \code{1} less than \code{n}. 

Make changes to \code{main} so that it will print the result of calling \code{sum-1-to-n} instead of \code{factorial}.  Save it as \code{sum12n.mer} (or any filename you want). You can run it just like you did \code{factorial.mer}:

\begin{REPL}
$ java -jar $MERUEM_HOME/meruem.jar sum12n 5
15
$ java -jar $MERUEM_HOME/meruem.jar sum12n 10
55
$ java -jar $MERUEM_HOME/meruem.jar sum12n 70
2485
$ java -jar $MERUEM_HOME/meruem.jar sum12n 0
0
$ java -jar $MERUEM_HOME/meruem.jar sum12n -3
0
\end{REPL}

Works fine! Now what if we try inputting a bigger number:

\begin{REPL}
$ java -jar $MERUEM_HOME/meruem.jar sum12n 400
Exception in thread "main" java.lang.StackOverflowError
	at meruem.Evaluate$.apply(Evaluate.scala:15)
	at meruem.Evaluate$.apply(Evaluate.scala:27)
.........
\end{REPL}

Tadaah! It's an error. More specifically, it's a \keyword{stackoverflow error}. Allow me to explain the issue. 

Every recursive call we make requires the allocation of a new \keyword{stack frame}, which means that the deeper the recursion gets, the more \keyword{stack frames} are needed. The problem is, we are only provided with a limited number of stack frames. So, if the recursion requires more stack frame than necessary, you'd get a \code{StackOverflowError}. Here's what the stack looks like when we execute \code{(sum-1-to-n 5)}:

\begin{REPL}
(sum-1-to-n 5)
(+ 5 (sum-1-to-n 4))
(+ 5 (+ 4 (sum-1-to-n 3)))
(+ 5 (+ 4 (+ 3 (sum-1-to-n 2))))
(+ 5 (+ 4 (+ 3 (+ 2 (sum-1-to-n 1)))))
(+ 5 (+ 4 (+ 3 (+ 2 (+ 1 (sum-1-to-n 0))))))
(+ 5 (+ 4 (+ 3 (+ 2 (+ 1 0)))))
(+ 5 (+ 4 (+ 3 (+ 2 (+ 1)))))
.....
(+ 5 10)
15
\end{REPL}

Now, that's a pretty exhausting! Imagine if we tried tracing the stack of \code{(sum-1-to-n 400)}. 

In the next section I am going to show you how to fix this kind of problem.