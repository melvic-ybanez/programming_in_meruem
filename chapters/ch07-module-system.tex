When your program gets really big and complicated, it is not a very good idea to put everything into one source file. You need to split the program into smaller pieces, and make each of these pieces as reusable as possible so different parts of the program can use it. Such a piece is called a \keyword{module}.

\section{What is a Module?}
A \ikeyword{module} in Meruem is a reusable named component of a program, just like a function. One difference between a module and a function is that a function contains expressions while a module can contain functions. In other words, a large program can be composed of several modules, and each of these modules can be composed of functions. One source file serves as a single module, and the functions and variables within that source file are the members of that module. From now on, we are going to use the term "module" instead of "source file" when we refer to Meruem files.

In other to use a module you need to \keyword{import} it. You can do that using the \code{import} command. The \code{import} command takes the name of the module to import as a string argument. Let us create a simple module that has one variable and one two functions:

\begin{Meruem}
(def x 10)

(defun squared (x) (* x x))

(defun cubed (x) (* x x x))
\end{Meruem}

Save it as \code{module\_demo.mer}. The next section will show how to import it.

\section{Import a module in the REPL}
Fire up the REPL and import the module using the \code{import} command as follows:

\begin{REPL}
meruem> (import "module_demo")
SomeModule(module_demo, MutableList(/home/melvic/meruem/lib/prelude), ArrayBuffer(cubed, module, x, squared)})
\end{REPL}

Note that you don't include the file extension when specifying the name of the module to import. The printed result shows information about the module being imported, like its name, path in the file system, and members. The fact that such information was displayed means that the import was successful. If you try to import a module that doesn't exist, you'll get an error:

\begin{REPL}
meruem> (import "modulo")
An error has occurred. File Not Found: /home/melvic/meruem/lib/modulo
Source: .home.melvic.meruem.lib.prelude [0:0}]
<undefined position>
\end{REPL}