In this chapter, we are going to make use of the REPL only. We are not going to use Winter (or any text editor you have right now). If you haven't already installed the REPL, please go back to section~\ref{sec:installing-meruem} and follow the instructions on how to install Meruem before proceeding. Remember, the Meruem distribution is already bundled with a REPL.

\section{Data Types}
Programming always involves manipulating data. For instance, writing a program that adds two random numbers involves working on numbers. Reading the contents of a file involves the manipulation of files and strings. An enrolment system requires the presence of data that represent the student information, the class schedules, and others. Even the \code{Hello World} program that we wrote earlier wouldn't even be completed if we didn't know what data to print to the screen. Whatever it is you want to do, you need some data.

Now, the thing about data is they don't all have the same classifications, and the operations that you can perform on a data depend on the classification of that data. For example, you can add a number to another number but you can't add a number to a student information. (That wouldn't really make sense.) This classification of data is known as a \keyword{data type}.

A \ikeyword{data type} tells you how a thing is classified, what set of values belongs to this type, and what operations can be performed on it. Meruem has a short list of supported data types. Let's discuss each of them, starting with the \code{Number} types.